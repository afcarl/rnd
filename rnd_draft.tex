% arara: pdflatex
% arara: biber
% arara: pdflatex
% arara: pdflatex
%\documentclass[12pt, a4paper, parskip=half, listof=totoc, bibliography = totoc]{scrartcl}
\documentclass[a4paper, 11pt, listof=totoc, bibliography=totoc]{scrartcl}

\usepackage[english]{babel} 
\usepackage[T1]{fontenc} %latex Ausgabefont
\usepackage[utf8]{inputenc} %Eingabecodierung
%\usepackage{lmodern}

% bibliography
\usepackage[pdfborderstyle={/S/U/W 0}]{hyperref}
\usepackage[hyperref = true, backend=biber, style=numeric, sorting = none]{biblatex}
\addbibresource{bibliography.bib}
\addbibresource{bibliography_related.bib}
\AtEveryBibitem{\clearfield{note}}
\AtEveryBibitem{\clearfield{timestamp}}
\AtEveryBibitem{\clearfield{urldate}}
\AtEveryBibitem{\clearfield{doi}}
\AtEveryBibitem{\clearfield{isbn}}
\AtEveryBibitem{\clearfield{series}}
\AtEveryBibitem{\clearfield{volume}}
\AtEveryBibitem{\clearfield{number}}
\AtEveryBibitem{\clearfield{pages}}
\AtEveryBibitem{\clearfield{issue}}
\AtEveryBibitem{\clearfield{publisher}}
\DeclareSourcemap{
  \maps[datatype=bibtex, overwrite]{
    \map{
%      \step[fieldset=language, null]
      \step[fieldset=urldate, null]
    }
  }
}
%---Packages---
\usepackage{graphicx}
\usepackage{float}
\usepackage[format = hang, figurename = {Figure}, tablename = {Table}, font = small, labelfont= it, position = below]{caption}
\usepackage[format = hang, indention = 0.0cm]{subcaption}
\usepackage{url}
\usepackage{tabularx}
\usepackage{textcomp}
\usepackage{amsmath}
\usepackage{amsfonts}
\usepackage{wrapfig}
\usepackage{enumitem}
\usepackage{parskip}
\usepackage{setspace}


\begin{document}
\thispagestyle{empty}
\begin{center}
\large
\textbf{Research \& Development Report} \\
Master Autonomous Systems \\
University of Applied Sciences Bonn-Rhein-Sieg \\
\LARGE

\vspace{1cm}

\vspace{-0.75\baselineskip}\rule{\linewidth}{1.5pt}
\textbf{Evaluation of current Approaches for Situation-Awareness in Autonomous Systems from Action Recognition in Video Data} \\
\vspace{-0.25\baselineskip}\rule{\linewidth}{1.5pt}

\vspace{1cm}

\normalsize
\textit{Author:}\\
Maximilian Schöbel\\
Mat. Nr. 9027059
\bigskip

\textit{Advisors:}\\
Prof. Dr. Erwin Prassler\\
Prof. Dr. Paul G. Plöger\\
\bigskip

January 12th, 2017

\vfill

\textbf{Abstract}
\end{center}
\small
Situation awareness is an abstract concept, which includes several independent manifestations and sensory inputs.
An important aspect of situation awareness is the knowledge of actions, that are performed by persons in the vicinity of an agent, to derive a suitable policy for the agent's own future actions.
This report investigates vision-based human action recognition from video data as a key step towards the understanding of scenes by autonomous (robotic) systems.
Recognition of actions furthermore enables applications in surveillance systems, patient monitoring, human-computer interfaces, assisted living environments and AI.
So far, computer vision research produced two main approach classes of machine learning based algorithms for classifying actions in video-data:
Either using arbitrary classification techniques on top of hand-crafted feature extractors or end-to-end trainable deep learning architectures.
This report provides a detailed review of current deep learning approaches for video-based action recognition in addition to a comparison with conventional hand-crafted feature approaches.
Furthermore, publicly available video datasets are reviewed, since the availability of labeled large-scale datasets is a big challenge in deep learning.
Even very recent video-datasets do not match the size of existing image datasets.
Results indicate, that deep learning techniques are competitive but not yet able to significantly outperform conventional hand-crafted feature methods in video-based action recognition.

\newpage

\thispagestyle{empty}
\hfill

\newpage

\thispagestyle{empty}
\footnotesize
\tableofcontents
\normalsize

\newpage

\section*{Statement of Originality}

This is to certify that the content of this report is my own work and that all sources have been acknowledged.


Date:


Signature: 
\newpage
\hfill
\newpage

Abstract:
We investigate human action recognition from video data as a key step towards the autonomous understanding of scenes in the environment of an autonomous (robotic) system.
Describe current approaches.
Biggest problems: Availability of data
How to overcome: Unsupervised learning


\section{Introduction}
META: Brief but concise review of the first two "W"s.

The operation of autonomous mobile systems in public, uncontrolled environments is despite active research still a difficult task.

Humans are perfectly able to act and move in unknown, crowded environments and even react successfully to new situations because they are aware of their surroundings.

An important part of Situation Awareness is the knowledge of what actions are currently performed by persons in the vicinity of an agent.
This knowledge enables the agent to derive a suitable policy for its own future actions.

Actions of interest are single-person actions, person-person interactions, person-object interactions and group activities.

Enabling situation-awareness in autonomous systems is an important goal, which has an impact on other problems in autonomous systems.

Possible applications:\\
Pedestrian movement predicition in robotics,\\
Risk and danger evaluation through video surveillance in public environements,\\
surveillance of children or the elderly in assisted living environments,\\
patient monitoring in hospitals,\\
video retrieval (content-based video indexing),\\
human-computer interaction.\\

Requirement: Automated recognition of high-level actions.

Situation awareness is an abstract concept, which includes lots of independent manifestations and involves multiple sensory inputs.

This work focuses on approaches that process time-sequential video data, because video-cameras represent a cost-effective and widely used technology in many existing systems.

Motivation for using videos: Promising results in classfication tasks from images.
Video adds another (temporal) dimension, which conveys a lot of information that can be accessed for classification as well.
Video provides natural data augmentation cite:simnoyan two-stream paper (??)

\subsection{Situation Awareness from video data}

META: General Definition of Situation Awareness in the context of autonomous systems.

Placement of Action Recognition among other vision-based methods, i.e.

Action Prediction, Anomaly Detection, Event and Action Detection, Person/Pedestrian Detection, Gesture Recognition.o

Abnormal event detection O. Boiman and M. Irani. Detecting irregularities in images and in video. IJCV, 2007

Activity understanding ``activity forecasting''

Definitions of the above methods.

Simple case: Video contains the performance of a single human action which needs to be classified into one of several preknown classses.

General real-world case: System operates on a video stream and needs to perform continuous recognition of human actions, including detection of beginning and endings times of containing acions.

General Processing Pipeline for Action Recognition: Person Detection -> Tracking -> Action Detection -> Segmentation -> Action recognition.

Action Recognition: A part of Computer Vision research, it's goal is to automatically analyze human actions/actitvities from video-data.

Other sensory input than video possible 

\subsection{The Action Recognition Problem}
Action Recognition is a classification-task.

Difference to face/gate recognition: Generalize over person characteristics.

Intra- and inter-class variances.

Background and recording settings.

Temporal variations.

Obtaining and labeling training data.

Main task of action recognition research: Overcome these challenges and built systems, that recognize actions robustly, even when performed by different persons in differently lighted environments at different speeds.

Main components: (i) A discriminative architecture that is able to recognise the general characteristics of different action classes while ignoring personal characteristics of different performers.
(ii) Large datasets that provide this information by containing many different examples for each action class.

\subsection{Survey Papers in Action Recognition (Related work)}

Review of most important/recent review papers in Action Recognition with traditional and Deep Learning approaches.

\subsubsection{A survey on vision-based human action recognition, Ronald Poppe (2010)}

\textbf{Definition of action:} Uses the hierarchical classification of human motion in action primitives, actions and activities as given in Moeslund et al. (cite ??)

Action primitives are atomic movements at the limb-level.

Actions are possibly cyclic whole body movements and consist of multiple action primitives.

Activities consist of multiple actions whose subsequent execution make the movement interpretable.

Example: Action primitives: Left/right leg forward -> Action: Starting, Running, Jumping -> Activity: Jumping hurdles.

\textbf{Scope:} Gives a very good classification of conventional methods in human action recognition.

The discussion is split according to video representations and classification methods.

Challenges of the domain are described very well.

\textbf{Deficits:} No Deep Learning methods are discussed.

Datasets and benchmarks are only discussed briefly.

\subsubsection{Human Activity Analysis: A Review -- Aggarwal and Ryoo (2011)}

Gives an approach-based taxonomy.

\subsubsection{A survey on vision-based methods for action representation, segmentation and recognition -- Weinland et al. (2011)}

\subsubsection{A survey of video datasets for human action and activity recognition -- Chaquet et al. (2013)}

\subsubsection{A review of unsupervised feature learning and deep learning for time-series modeling -- Längkvist et al. (2014)}

\subsubsection{Going Deeper into Action Recognition: A survey -- Herath et al. (2016)}

\textbf{Definition of action:} 

\subsection{Important aspects of Action Recognition Approaches}
video representations, i.e.\ how to apply fixed length model to variable length videos.

\section{Conventional Methods in Action Recognition}
META: 
Condensed overview and description of conventional Methods in action Recognition using the taxonomy of Aggarwal and Ryoo's fine survey paper.
More detailed description of methods using local-features, since these have become the standard approach in action recognition after Aggarwal and Ryoo's overview.

\begin{figure}[H]
    \centering
    \includegraphics[width=\textwidth]{img_conventional/taxonomy_conventional_methods.png}
    \caption{Approach-based taxonomy for conventional methods in human activity recognition as given by Aggarwal and Ryoo\cite{aggarwal_human_2011}}
    \label{fig:conventional_taxonomy}
\end{figure}

3 Main components in action recognition using local features: Feature Extraction, Representation Building, Classification.

Methods for feature extraction: Interest point detectors or dense sampling.

Space-time interest point detectors: Harris3D\cite{laptev_space-time_2005}, Cuboids\cite{dollar_behavior_2005}, Hessian Detector\cite{willems_efficient_2008}

Descriptors for 3D volumes around previously detected space-time interest points: Histogram of Gradient HOG\cite{dalal_histograms_2005-1}, Histogram of Optical Flow (HOF)\cite{laptev_learning_2008}, 3D Histogram of Gradient (HOG3D)\cite{klaser_spatio-temporal_2008}, Extended SURF (ESURF)\cite{willems_efficient_2008}

``standard approach to video classification'' described in \cite{karpathy_large-scale_2014}

\subsection{Overview}

\subsection{Local Features}

\subsubsection{Feature extractors}

\subsubsection{Aggregation Methods}

\textbf{Bag of visual Words paradigm}

\textbf{Fisher Vector}


\section{Deep Learning Methods in Action Recognition}
Review of approaches that use Deep Learning.

\subsection{Spatio-Temporal Networks}
I.e. convolutional methods.

\subsection{Multiple Stream Networks}
The most successfull architecture at action recognition. They are equally powerful as the improved dense trajectories approach. cite TDD

\subsection{Generative Models}
Restricted Boltzmann Machine

\subsection{Temporal Coherency Networks}

\section{Datasets and Benchmarks in Action Recognition}

\subsection{Review of Datasets for Human Action Classification}
Review of the most important currently existing datasets, focus on newest ones (since 2013)

Reference dataset survey paper.

\section{Data Augmentation}<++>

\subsection{Alternative Benchmarks for Action Recognition Algorithms}

\subsection{Inter-Dataset Approaches}


\section{Comparison and Evaluation}
\label{sec:evaluation}
Using the analysis of conventional and deep learning methods in action recognition, this section provides a comparison and evaluation of the presented methods.
Results of deep learning approaches, which were reviewed in this report, are given in the following table \ref{tab:deep_results}.

\begin{minipage}[b]{.03\linewidth}
\setstretch{1.25}
\scriptsize
\cite{ji_3d_2013}\\
\cite{baccouche_sequential_2011}\\
\cite{karpathy_large-scale_2014}\\
\cite{tran_learning_2015}\\
\cite{varol_long-term_2016}\\
\cite{simonyan_two-stream_2014}\\
\cite{ng_beyond_2015}\\
\cite{wang_towards_2015}\\
\cite{feichtenhofer_convolutional_2016}\\
\cite{wang_action_2015}\\
\cite{srivastava_unsupervised_2015}
\cite{palasek_action_2016}
\cite{misra_shuffle_2016}
\\\\
\end{minipage}
\begin{minipage}[b]{.96\linewidth}
\begin{table}[H]
    \centering
    \includegraphics[width=\textwidth]{img_evaluation/deep_results}
    \caption{Best reported action recognition accuracies of reviewed deep learning approaches.}
    \label{tab:deep_results}
\end{table}
\end{minipage}

We observe from the distribution of used benchmarking datasets, that UCF-101 and HMDB-51 can be seen as the current de facto standard benchmarks in action recognition, although UCF-101 is considered extremely small \cite{wang_towards_2015}.
Judging by the best performance on UCF-101, the approaches of \textcite{wang_towards_2015}, \textcite{feichtenhofer_convolutional_2016} and \textcite{wang_action_2015} perform best.
These approaches base on 2D convolutional neural network architectures, that were initially designed for image processing tasks and take advantage of pre-training on large-scale image datasets.
\cite{wang_towards_2015} and \cite{feichtenhofer_convolutional_2016} utilise deep ConvNets in an two-stream setup.
\cite{wang_action_2015} apply a hybrid conventional-deep approach by using a two-stream ConvNet as generic feature extractor along densely sampled trajectories.

Among the approaches, which explicitly incorporate the spatio-temporal structure of videos in the used architecture by implementing 3D convolutions, the \textit{C3D} approach of \textcite{tran_learning_2015} performs best.
\textcite{tran_learning_2015} use a 3D ConvNet which is deeper than comparable approaches (three additional convolutional layers) and leverage pre-training on the Sports-1M dataset.
Since \textit{C3D} is designed for reusability on different vision tasks, an implementation was made publicly available in the Caffe framework \cite{jia_caffe:_2014}.

Given these results it can be derived, that deeper architectures and bigger datasets improve performance in action recognition from video using deep learning approaches.
This has also been observed in the area of object recognition from still images \cite{simonyan_very_2014, szegedy_going_2015, he_deep_2015}.

Table \ref{tab:conventional_results} shows the performance of state-of-the-art hand-crafted feature methods in action recognition.

\begin{minipage}[b]{.03\linewidth}
\setstretch{1.3}
\scriptsize
\cite{wang_action_2013}\\
\cite{wang_action_2013}\\
\cite{cai_multi-view_2014}\\
\cite{lan_beyond_2015}\\
\\
\end{minipage}
\begin{minipage}[b]{.96\linewidth}
\begin{table}[H]
    \centering
    \includegraphics[width=\textwidth]{img_evaluation/conventional_results}
    \caption{Action recognition accuracies of state-of-the-art hand-crafted feature methods.}
    \label{tab:conventional_results}
\end{table}
\end{minipage}

The results on UCF-101 show: Although several deep learning based approaches outperform conventional hand-crafted feature methods, the latter nonetheless yield competitive results, especially the approach of \textcite{lan_beyond_2015}.
Reasons for this can be found in the sizes of available action recognition video datasets.
Table \ref{tab:benchmarks_results} provides a comparison of currently available datasets.

\begin{table}[H]
    \centering
    \includegraphics[width=0.7\textwidth]{img_evaluation/benchmarks_results}
    \caption{Size comparison of reviewed benchmarking datasets.}
    \label{tab:benchmarks_results}
\end{table}

Although the Sports-1M dataset is roughly 100 times bigger than UCF101, most approaches rely on the smaller UCF101 or HMDB51 for action recognition, because of the label noise in YouTube-1M \cite{feichtenhofer_convolutional_2016}.
There is a noticeable trend towards larger datasets however.
The very recently released YouTube-8M dataset is roughly eight times bigger than YouTube-1M, but still does not match the size of common image datasets (ImageNet contains $14,197,122$ images \cite{_imagenet_????})
The difficulty to obtain large-scale video datasets originates in videos being more difficult to store and annotate than images \cite{karpathy_large-scale_2014}.

Given its recency, the YouTube-8M has not been used widely in the action recognition community.
Considering its unmatched size in video datasets so far, the creators of the set note, that it will most likely improve the performance of deep learning approaches in action recognition \cite{abu-el-haija_youtube-8m:_2016}.

Table \ref{tab:adl_results} summarizes the characteristics of currently available datasets, that contain videos of the daily-living.

\begin{table}[H]
    \centering
    \includegraphics[width=0.7\textwidth]{img_evaluation/adl_results}
    \caption{Size comparison of reviewed daily-living datasets.}
    \label{tab:adl_results}
\end{table}

The Charades dataset \cite{sigurdsson_hollywood_2016} and ActivityNet \cite{caba_heilbron_activitynet:_2015} provide comparably large collections of ADL videos.
Especially the Charades dataset has the advantage of containing videos, that were collected in real-world environments by regular people.
In order to implement an action recognition system in an assisted living environment, we recommend applying the \textit{C3D} approach \cite{tran_learning_2015} to one or possibly both of these datasets.
The advantages of \textit{C3D} are:
\begin{enumerate}
    \item It is the deepest model implementing 3D convolutions.
    \item A model pre-trained on Sports-1M is publicly available.
    \item It was designed to be repurposed on different vision tasks.
\end{enumerate}

\newpage
\section{Conclusion and Future Directions}
This report provides a detailed review of current deep learning approaches in human action recognition from video along with a comparison to conventional hand-crafted feature based approaches.
Results show, that deep learning yields state of the art performance, but conventional hand-crafted feature based methods still perform competitive.
This has been observed in the action recognition community as well \cite{varol_long-term_2016}.

Two reasons for this phenomenon can be identified:
\begin{enumerate}
    \item Current deep learning architectures in video action recognition are shallow compared to their image based counterparts \cite{wang_towards_2015}, because videos are higher dimensional than images and processing them is computationally more expensive.
    \item Available action recognition datasets are too small, since videos are more difficult to store and annotate than images \cite{karpathy_large-scale_2014}\cite{simonyan_two-stream_2014}\cite{wang_towards_2015}
\end{enumerate}

Considering the challenge of obtaining sufficient training data for deep learning architectures, approaches that leverage data from more than one dataset can be identified as future directions of research:
\begin{itemize}
    \item \textbf{Unsupervised Pre-training} as presented in section \ref{sec:generative} of this report and \textcite{varol_long-term_2016}.
    \item \textbf{Transfer Learning} was conducted by \textcite{karpathy_large-scale_2014}.
    \item \textbf{Multi-task Learning} as implemented by \textcite{simonyan_two-stream_2014}.
\end{itemize}

%\subsubsection{Transfer learning}

%See Karparthy 'Large-scale video classification with convolutional neural networks' 2014

%A. S. Razavian, H. Azizpour, J. Sullivan, and S. Carls-
%son. CNN features off-the-shelf: an astounding baseline for
%recognition

%N. Zhang, M. Paluri, M. Ranzato, T. Darrell, and L. Bourdev. Panda: Pose aligned networks for deep attribute modeling. In CVPR, 2014.
%
%B. Zhou, A. Lapedriza, J. Xiao, A. Torralba, and A. Oliva. Learning deep features for scene recognition using places database. In NIPS, 2014.


\newpage

\printbibliography

\end{document}
