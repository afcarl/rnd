\section{Introduction}
Brief but concise review of the first four "W"s.

Applications: security-driven video surveillance in public environements, surveillance of children or the elderly in assisted living environments, patient monitoring, video retrieval (content-based video indexing) , human-computer interaction, robotics.

Requirement: Automated recognition of high-level actions.

Because of large variations in used datasets and evaluation practice, approaches are discussed conceptually.

Focus on Deep Learning approaches, since conventional methods have been surveyed in many articles before (see related work section).


\subsection{Situation Awareness from video data}
General Definition of Situation Awareness in the context of autonomous systems.

Placement of Action Recognition among other vision-based methods, i.e. Action Prediction, Anomaly Detection, Event and Action Detection, Person/Pedestrian Detection, Gesture Recognition.

Definitions of the above methods.

Simple case: Video contains the performance of a single human action which needs to be classified into one of several preknown classses.

General real-world case: System operates on a video stream and needs to perform continuous recognition of human actions, including detection of beginning and endings times of containing acions.

General Processing Pipeline for Action Recognition: Person Detection -> Tracking -> Action Detection -> Segmentation -> Action recognition.

Action Recognition: A part of Computer Vision research, it's goal is to automatically analyze human actions/actitvities from video-data. 

Other sensory input than video possible: 


\subsection{Survey Papers in Action Recognition (Related work)}

Review of most important/recent review papers in Action Recognition with traditional and Deep Learning approaches.

\subsubsection{A survey on vision-based human action recognition, Ronald Poppe (2010)}

\textbf{Definition of action:} Uses the hierarchical classification of human motion in action primitives, actions and activities as given in Moeslund et al. (cite ??)

Action primitives are atomic movements at the limb-level.

Actions are possibly cyclic whole body movements and consist of multiple action primitives.

Activities consist of multiple actions whose subsequent execution make the movement interpretable.

Example: Action primitives: Left/right leg forward -> Action: Starting, Running, Jumping -> Activity: Jumping hurdles.

\textbf{Scope:} Gives a very good classification of conventional methods in human action recognition.

The discussion is split according to video representations and classification methods.

Challenges of the domain are described very well.

\textbf{Deficits:} No Deep Learning methods are discussed. 

Datasets and benchmarks are discussed only very briefly.

\subsubsection{Human Activity Analysis: A Review -- Aggarwal and Ryoo (2011)}

Gives an approach-based taxonomy.

\subsubsection{A survey on vision-based methods for action representation, segmentation and recognition -- Weinland et al. (2011)}

\subsubsection{A survey of video datasets for human action and activity recognition -- Chaquet et al. (2013)}

\subsubsection{A review of unsupervised feature learning and deep learning for time-series modeling -- Längkvist et al. (2014)}

\subsubsection{Going Deeper into Action Recognition: A survey -- Herath et al. (2016)}

\textbf{Definition of action:} 

\subsection{Challenges in Action Recognition}
Action Recognition is a classification-task.

Intra- and inter-class variances.

Background and recording settings.

Temporal variations.

Obtaining and labeling training data.

Difference to face/gate recognition: Generalize over person characteristics.

Main task of action recognition research: Overcome these challenges and built systems, that recognize actions robustly, even when performed by different persons in differently lighted environments at different speeds.
